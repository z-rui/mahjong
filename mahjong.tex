\input luaotfload.sty
\input luatypezh
\input mjmac

\normalbaselineskip=14.5pt
\normalbaselines

\centerline{麻将规则概述}

\beginsection 1. 牌

麻将牌分为数牌和字牌两类。 数牌有万、 索、 饼三种, 上有不同的花样。
每张数牌对应一至九的一个数字, 以花样的个数或以汉字表示。
$$\matrix{
\strut&1&2&3&4&5&6&7&8&9\cr
万&🀇&🀈&🀉&🀊&🀋&🀌&🀍&🀎&🀏\cr\noalign{\medskip}
索&🀐&🀑&🀒&🀓&🀔&🀕&🀖&🀗&🀘\cr\noalign{\medskip}
饼&🀙&🀚&🀛&🀜&🀝&🀞&🀟&🀠&🀡\cr
}$$
字牌包括风牌 ($🀀🀁🀂🀃$) 和三元牌 ($🀄🀆🀅$)。
因此, 共有 34~种不同的牌面。 每种牌面有 4~张, 因此, 共有 136~张牌。

数字为 1 或 9 的牌, 以及字牌, 统称幺九牌。

同花样的三张连续数牌称为顺子; 三张相同的牌称为刻子;
四张相同的牌在杠牌 (见~2.2~节) 后, 称为杠子。
顺子、 刻子、 杠子统称面子。 两张相同的牌构成对子。
一般而言, 和牌需要四个面子和一个对子, 这里的对子被称为雀头。

\beginsection 2. 局

每局从配牌开始, 到有人和牌或者流局结束。

开局时, 所有的牌被打乱, 背面向上, 并堆成牌山。
牌山在桌上摆成一圈, 且分上下两层。 叠在一起的上下两张牌被称为一墩。
牌山的末尾 7~墩 (14~张) 牌称为王牌, 不被摸取。
最后两墩牌称为岭上牌, 用示在开杠时补牌。
倒数第三墩的上张牌在开局时被翻开作为宝牌指示牌。
其余墩的上张牌在杠牌时被翻开, (翻开后) 也作为宝牌指示牌。

配牌时, 其中一人从牌山中取得 14~张牌, 称为东家,
其余按逆时针依次为南家、 西家、 北家, 每家 13~张牌。
每家手上有的牌称为手牌, 除副露 (见~2.2~节) 外, 只有自己可以看到。

东家先将手上的一张牌打出。 然后, 按逆时针次序轮流进行摸打。
即, 先从牌山中摸一张牌, 然后在手牌中选一张打出。
被打出的牌应整齐摆在各自面前, 称为牌河。
每一轮摸打称为一巡。

\subsection 2.1. 和牌

当自己的手牌可以和牌, 或者别家打出了一张加入自己手牌即可和牌的牌时,
可以宣告和牌。 前一种情况称为自摸, 后一种情况称为荣和。
和牌的条件为:
\item{1.} 牌型符合四个面子加雀头的形式, 或者特殊牌型 (见第~4~节)。
\item{2.} 有役 (见第~4~节);
\item{3.} 荣和时不存在振听 (见第~2.6~节) 的情况。

\subsection 2.2. 鸣牌

鸣牌分为吃、 碰、 杠这几种情况。
多人鸣牌或和牌时, 和牌优先于鸣牌, 碰/杠优先于吃。

当别家打出的牌可以和手中的牌构成刻子或杠子时,
可以宣布碰牌或杠牌 (这时杠牌也称明杠)。
碰牌后, 需要把构成的刻子公开, 并放置在右手侧, 称为副露。
从别家获得的牌需要横放, 并靠左/中/右放置, 表示该牌来自上/对/下家,
如 $\sym 🀀🀀\rotl{🀀}$ (碰入的$🀀$来自下家) 或 $\sym 🀟\rotl{🀟🀟}🀟$ 
($🀟$来自对家)。

当上家打出的牌可以和手中的牌构成顺子时, 可以宣布吃牌。
构成的顺子也要副露。

如手牌中有一张和已经副露的刻子相同的牌, 可以宣布加杠,
并把这张牌横放在已经横放的牌上。
当自已的手牌中含有四张相同的牌时, 可以宣布暗杠。
暗杠时将两侧两张牌背朝上, 如 $🀫🀀🀀🀫$。

吃牌或碰牌后, 需打出手中的一张牌。
不允许食替 (吃入一张牌后, 打出可以和顺子中的另外两张牌构成顺子的牌)。
例如, 吃$🀊$构成顺子 $🀊🀋🀌$~时, 不允许立即打出$🀊$或$🀍$。
杠牌后, 需要先在岭上牌中摸取一张牌, 再将手牌中的一张打出,
然后翻开王牌中的另一张宝牌指示牌。
以上步骤完成后, 由鸣牌者的下家继续摸打。
注意: 碰牌和明杠有可能打乱原先的摸打次序。

\subsection 2.3. 流局

流局分为荒牌和中途流局两种情况。

当牌山中除王牌外的牌全部被摸完时, 没有更多的牌可供摸打。
此时宣告流局 (荒牌), 本局结束。

当发生下列情况时, 宣告中途流局, 本局提前结束:
\item{1.} 九种九牌: 某人打出第一张牌前, 如果手牌中有至少 9~张幺九牌,
且此前未有人鸣牌, 则这时他有权利宣告中途流局;
\item{2.} 四风连打: 第一巡中, 四家连续打出同一张风牌;
\item{3.} 四杠散了: 一局中 (非同一人) 开出了四个杠子;
\item{4.} 四家立直: 四家都宣告了立直 (见~2.5~节);
\item{5.} 三家荣和: 某人打出的一张牌造成三家同时和牌。

\subsection 2.4. 听牌

距离和牌牌型仅差一张牌的状态, 称为听牌。
对于四个面子加雀头的牌型, 听牌分为如下几种:
\item{1.} 两面听: 手牌中已有连续的两张数牌,
需要另一张构成顺子, 且有两种选择。
例如 $$\hbox{$🀫🀫🀫\,🀫🀫🀫\,🀫🀫🀫\,🀌🀍\,🀫🀫$ 听 $🀋🀎$}$$
\item{2.} 边张听: 类似, 但是因为两张数牌中有一张幺九牌, 所以只有一种选择。
例如 $$\hbox{$🀫🀫🀫\,🀫🀫🀫\,🀫🀫🀫\,🀗🀘\,🀫🀫$ 听 $🀖$}$$
\item{3.} 嵌张听: 已有顺子的头尾两张, 听中间一张。
例如 $$\hbox{$🀫🀫🀫\,🀫🀫🀫\,🀫🀫🀫\,🀛🀝\,🀫🀫$ 听 $🀜$}$$
\item{4.} 双碰听: 已有两个对子, 再得两者之一即可构成一个刻子。
例如 $$\hbox{$🀫🀫🀫\,🀫🀫🀫\,🀫🀫🀫\,🀀🀀\,🀂🀂$ 听 $🀀🀂$}$$
\item{5.} 单骑听: 剩一张单牌, 需要再取得一张构成雀头。
例如 $$\hbox{$🀫🀫🀫\,🀫🀫🀫\,🀫🀫🀫\,🀫🀫🀫\,🀐$ 听 $🀐$}$$

听牌与否, 仅靠牌面判定, 与是否可以和牌无必然联系。
如果因无役而不能和牌, 则称这类听牌为形式听牌。
如果所听的牌已经全被用完, 因而没有和牌的可能, 则称这类听牌为空听。

\subsection 2.5. 立直

在打出一张牌时, 如果自家没有吃、 碰和明杠, 且即将听牌, 则可以宣告立直。
宣告立直时打出的牌应横放。 如这张牌被别家荣和, 则立直不成立。
如果横放的牌被鸣牌, 则下一次打出的牌仍要横放。

宣告立直需交出 1000~点, 称为供托, 归之后和牌的人。
已宣布立直的玩家, 每次摸打只允许打出摸入的牌, 不可打出其他手牌,
也不允许鸣牌, 但可以暗杠。

\subsection 2.6. 振听

当所听的任一张牌满足下列条件时, 玩家将被禁止荣和, 只能通过自摸和牌。
这种状态称为振听。

\item{1.} 舍牌振听: 所听牌是自己曾经打出的牌;
\item{2.} 同巡振听: 所听牌是一巡之内别家打出的牌;
\item{3.} 立直振听: 在立直的情况下, 所听的牌是自己或别家曾打出的牌。

\beginsection 3. 场

四局构成一场。 每一场有一张风牌与之对应, 称为场风牌。
东风场的四局分别称为东一局、 东二局等, 其余各场依此类推。
从东风场开始, 依据规则不同, 进行若干场比赛:

\item{1.} 东风战: 仅东风场四局;
\item{2.} 半庄战: 东风场、 南风场, 共八局;
\item{3.} 全庄战: 东、 西、 南、 北风场, 共十六局。

每局结束时, 满足以下任一种情况, 这些情况下, 局名不变, 原东家继续做东
(称为连庄)。

\item{1.} 东家和牌;
\item{2.} 荒牌, 且东家听牌;
\item{3.} 中途流局。

否则, 进入下一局, 并由原东家的下家做东家。
即, 东一局后进入东二局, 东四局后进入南一局, 其余依此类推。

东家和牌或者流局时, 下一局本场数加一, 且按照本场数将局次称为
“某某局一本场”、 “二本场”, 等等。
一旦东家以外的人和牌, 则本场数归零。

\beginsection 4. 役

当牌型或和牌方式满足一定条件时, 可获得役。 无役则不能和牌。
这些不同的条件称为各役种。 每个役种对应一个翻数, 称为
“一翻役”、 “二翻役”, 等等。

如有吃、 碰或明杠, 则标记★的役 (称为 “门前役”) 不成立,
而标记▲的役 (称为 “食下役”) 翻数减一。

\subsection 4.1 一翻役

\yaku*立直
宣告立直后和牌。

\yaku*一发
宣告立直后, 一巡之内荣和或自摸。 如期间发生鸣牌, 则此役不成立。

\yaku*门前自摸
没有吃、 碰、 明杠的情况下自摸和牌。

\yaku.断幺九
和牌牌型中不含幺九牌。

\yaku.役牌
含有役牌的刻子或杠子。 役牌包括: 场风牌、
自风牌 (东家的自风牌为$🀀$, 依此类推)、 三元牌。
有多组役牌的, 此役可以累计。
如果场风牌和自风牌是同一张牌 (称为连风牌), 按两组役牌计算。

\yaku*平和
三个顺子加一个非役牌构成的雀头, 听牌方式必须为两面听。

\yaku*一杯口
含有两个相同的顺子。

\yaku.岭上开花
杠牌后摸入的岭上牌造成的和牌。

\yaku.枪杠
别家加杠的牌和自己的手牌构成和牌。

\yaku.海底捞月
摸入牌山中最后一张可摸的牌 (称为海底牌) 时自摸和牌。

\yaku.河底捞鱼
别家摸入海底牌后打出的牌 (称为河底牌) 造成荣和。

\subsection 4.2 二翻役

\yaku^三色同顺
含有同数字的三种花色的顺子。

\yaku.三色同刻
含有同数字的三种花色的刻子 (含杠子)。

\yaku^一气通贯
含有同花色的 $123\,456\,789$ 三组顺子。

\yaku.对对和
四组刻子 (含杠子) 和雀头构成的和牌。

\yaku.三暗刻
和牌中有三组暗刻 (含暗杠)。
暗刻 (暗杠) 指没有借助别家打出的牌构成的刻子 (杠子)。

\yaku.三杠子
和牌中有三组杠子。

\yaku*七对子
含有七个不同的对子的特殊牌型。

\yaku^混全带幺九
面子和雀头均含有幺九牌, 且整副牌中含有字牌。

\yaku.混老头
所有牌均为幺九牌, 且整副牌中含有字牌。
不计混全带幺九, 但可以与对对和/七对子累计。

\yaku.小三元
含有三元牌的两组刻子和一个对子。

\yaku*双立直
打出第一张牌时就宣布立直。 宣布立直前如有人鸣牌, 则此役不成立。

\subsection 4.3 三翻役

\yaku^混一色
由同一花色的数牌和字牌组成的和牌。

\yaku^纯全带幺九
面子和将牌均含有幺九牌, 且整副牌不含字牌。

\yaku*二杯口
含有两组一杯口。 不计七对子、 一杯口。

\subsection 4.4 六翻役

\yaku^清一色
由同一花色的数牌组成的和牌。

\subsection 4.5 役满

役满的役是点数最高的役。 役满时, 无需计算翻数, 也不必考虑役满以下的役。

\yaku*国士无双
由 $🀇🀏🀐🀘🀙🀡🀀🀁🀂🀃🀄🀆🀅$ 加上任意一张幺九牌组成的特殊牌型。
此役允许抢暗杠。
如果听 13~张幺九牌, 则称为{\bf 国士无双十三面听}。

\yaku.大三元
含有三元牌的三组刻子 (含杠子)。

\yaku*四暗刻
含有四组暗刻 (含暗杠)。 如听牌方式为单骑听,
则称为{\bf 四暗刻单骑}。

\yaku.字一色
仅由字牌构成的和牌。

\yaku.绿一色
仅由 $🀑🀒🀓🀕🀗🀅$ 构成的和牌 (因这些牌牌面上只有绿色)。

\yaku.小四喜
含有风牌的三组刻子 (含杠子) 和一个对子。

\yaku.大四喜
含有风牌的四组刻子 (含杠子)。

\yaku.清老头
所有牌均为幺九牌, 且整副牌不含字牌。

\yaku*九莲宝灯
同花色的 1112345678999 加上该花色的任意一张数牌。
如果听该花色的 9~张数牌, 则称为{\bf 纯正九莲宝灯}。

\yaku.四杠子
含有四个杠子。

\yaku*天和
东家配牌时得到的 14~张牌构成和牌。

\yaku*地和
非东家在配牌后摸入第一张牌时自摸和牌。
如此前有人鸣牌, 则此役不成立。

\subsection 4.6 特殊役

\yaku.流局满贯
发生荒牌时, 如果自家打出的牌均为幺九牌, 且未被鸣牌, 则计作满贯。

\beginsection 5. 翻

和牌时, 根据役种和宝牌计算翻数。
除非另有说明, 役种的翻数可以复合。  役满则无需计算翻数。

宝牌指示牌的下一张牌被确定为宝牌, 按照以下顺序:
$$\let~\rightarrow\displaylines{
1~2~3~4~5~6~7~8~9~1\cr
🀀~🀁~🀂~🀃~🀀\cr
🀄~🀆~🀅~🀄\cr
}$$
开局 (杠牌) 时翻开的宝牌指示牌指定的牌称为 “表宝牌” (“杠宝牌”)。
如和牌者宣告了立直, 则每张翻开的宝牌指示牌所在牌墩的下张牌也翻开,
作为宝牌指示牌, 由它们指定的牌称为 “里宝牌” (“杠里宝牌”)。
有些对局中, 会将若干牌 (一般为$🀋🀔🀝$) 涂成红色, 这些牌本身就是宝牌,
也称为 “赤宝牌”。

除了役种带来的翻数外, 每张宝牌计一翻。
如果同一张牌多次成为宝牌, 则获得的翻数累计。

和牌时, 翻数达到 5~翻, 则计为{\bf 满贯}; 6~翻和 7~翻称为{\bf 跳满};
8---10~翻称为{\bf 倍满}; 11~翻和 12~翻称为{\bf 三倍满};
13~翻及以上称为{\bf 累计役满}。

\beginsection 6. 符

当和牌的翻数在满贯以下时, 需要计算牌面的符数。
符数由以下部分组成:

\item{1.} 符底, 计 20~符;
\item{2.} 门前荣和。 没有吃、 碰、 明杠的情况下荣和的, 计 10~符;
\item{3.} 面子加符。 非幺九牌, 明刻计 2~符, 暗刻计 4~符, 明杠计 8~符,
暗杠计 16~符; 幺九牌按照非幺九牌的两倍计符。
\item{4.} 听牌加符。 嵌张、 边张、 单骑听牌
(此处只考虑造成和牌的牌与手牌的关系) 计 2~符。
\item{5.} 雀头加符。 役牌作为雀头的, 计 2~符。
有些规则中, 连风牌作雀头计 4~符。
\item{6.} 自摸加符。 平和以外的役自摸, 计 2~符。

计算得到的符数以 10 为单位向上取整。
如果和牌是七对子, 则按 25~符固定符数计算, 不取整到 30~符。

\beginsection 7. 点

每位玩家初始拥有 25000~点, 每局结束时进行点数结算。
和牌者获得的点数由三部分组成:

\item{1.} 和牌点数, 根据役种、 翻数和符数确定;
\item{2.} 本场点数, 为本场数乘以 300;
\item{3.} 立直供托。

立直供托总是由宣告立直者支付。 其余前两项, 荣和时由放铳 (打出使人和牌的牌)
者支付, 自摸时由其余所有玩家支付。

计算和牌点数时, 设符数为~$x$, 翻数为~$y$。 然后计算基本点数
$$a=\cases{
\min\{x\times2^{2+y}, 2000\}& $y=1,2,3,4;$\cr
2000& $y=5;$ (满贯)\cr
3000& $y=6,7;$ (跳满)\cr
4000& $y=8,9,10;$ (倍满)\cr
6000& $y=11,12;$ (三倍满)\cr
8000& $y\ge 13.$ (累计役满)\cr
}$$
东家和牌时, 得 $6a$~点, 由放铳者支付, 或自摸时其余每人支付 $2a$~点;
其他玩家和牌时, 得 $4a$~点, 由放铳者支付, 或自摸时东家支付 $2a$~点,
另外两家各支付 $a$~点。
但是, 在支付的时候, 以 100~点为单位向上取整。

役满的情况, 按 $a=8000$ 计算。
一部分规则允许役满复合, 例如
$$\hbox{$🀀🀀🀀\,🀁🀁🀁\,🀂🀂🀂\,🀃🀃🀃\,🀆$ 和 $🀆$}$$
算作大四喜、 四暗刻单骑、 字一色, $a=24000$。
另有部分规则将下列役记作双倍役满, 按役满的两倍计点数。
\item{1.} 国士无双十三面听;
\item{2.} 四暗刻单骑;
\item{3.} 大四喜;
\item{4.} 纯正九莲宝灯。

也可从以下点数速查表中得到和牌点数。
格式为 ${\bf \underline{6a}/4a}\;(\underline{2a},a)$。
为节省版面, 表中的数已除以 100。

\midinsert\tabskip.587em plus .003em minus .003em\offinterlineskip
\def\mrule#1{\multispan{#1}\kern-\tabskip\hrulefill}
\halign to\hsize{\hfil\raise8pt\hbox{#}&&\vrule#&\hfil#\hfil\cr
\noalign{\hrule}
\omit&height10pt depth2.5pt&
1 翻&&2 翻&&3 翻&&4 翻&&满贯&&跳满&&倍满&&三倍满&&役满\cr
\noalign{\hrule}
\fu{20}{4}&&
\points{20}{40}{80}{120}&&
\points{30}{60}{120}{180}&&
\points{40}{80}{160}{240}&&
\points{60}{120}{240}{360}&&
\points{80}{160}{320}{480}\cr
\mrule{10}&& && && && &\cr
\fu{25}{4}&& && && && &\cr
\mrule{10}&& && && && &\cr
\fu{30}{4}&& && && && &\cr
\mrule{10}&& && && && &\cr
\fu{40}{3}&&\multispan3&& && && &\cr
\mrule8&    \multispan3&& && && &\cr
\fu{50}{3}&&\multispan3&& && && &\cr
\mrule8&    \multispan3&& && && &\cr
\fu{60}{3}&&\multispan3&& && && &\cr
\mrule8&    \multispan3&& && && &\cr
\fu{70}{2}&&\multispan5&& && && &\cr
\mrule6&    \multispan5&& && && &\cr
\fu{80}{2}&&\multispan5&& && && &\cr
\mrule6&    \multispan5&& && && &\cr
\fu{90}{2}&&\multispan5&& && && &\cr
\mrule6&    \multispan5&& && && &\cr
\fu{100}2&& \multispan5&& && && &\cr
\mrule6&    \multispan5&& && && &\cr
\fu{110}2&& \multispan5&& && && &\cr
\noalign{\hrule}
}\endinsert

如当前为 $x$~本场, 则荣和时放铳者另支付 $300x$~点,
自摸时其余各家各自支付 $100x$~点。

和牌时, 立直供托归和牌者。 如果是流局, 则供托保留。

发生荒牌时, 除非是荒牌满贯的情况, 需进行流局罚符。
设有 $x$~人听牌而 $y$~人未听牌,
则当 $xy\ne0$ 时, 未听牌者每人支付 $3000/y$~点, 听牌者每人获得 $3000/x$~点。

当所有牌局都打完, 或者有人点数为负数时, 比赛结束。
按照每人拥有的点数计算排名。
如计分, 则以 30000~点为界, 每超 1000~点得 1~分, 每不足 1000~点扣 1~分。
第一名额外得 20~分。
部分规则中, 如无人达到 30000~点, 则进入下一场
(如, 东风战东四局后要进入南风场), 直到有人达到 30000~点为止。
除此以外, 亦有其他的计分规则。

\bye
